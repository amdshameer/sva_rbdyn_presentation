\documentclass{beamer}

\usetheme{PaloAlto}
\usecolortheme{crane}

\title
{A short introduction to SVA and RBDyn}
\author
{Joris Vaillant}
\institute{LIRMM}{}
\date{Thursday 4 2014}
\subject{Spatial Vector Algebra, Rigid Body Physics}

\usepackage{listings}
\usepackage{graphicx}      % include this line if your document contains figures
\usepackage{amsmath} % assumes amsmath package installed
\usepackage{amsfonts} %% pour mathbb
\usepackage{upquote}

% vim: set fileencoding=utf8 :

\newcommand{\euclidspace}[1]{E^{#1}}
\newcommand{\motionspace}[1]{M^{#1}}
\newcommand{\forcespace}[1]{F^{#1}}

\newcommand{\motionvec}{\mathbf{m}}
\newcommand{\motionvecHat}{\hat{\motionvec}}
\newcommand{\motionvecOrigin}[1]{\motionvec_{#1}}
\newcommand{\motionvecHatOrigin}[1]{\motionvecHat_{#1}}

\newcommand{\motionvecDot}{\dot{\motionvec}}
\newcommand{\motionvecDotHat}{\dot{\motionvecHat}}

\newcommand{\motionvecFrame}[1]{{^{#1}\mathbf{m}}}
\newcommand{\motionvecHatFrame}[1]{{^{#1}\hat{\motionvec}}}

\newcommand{\velocityvec}{\mathbf{v}}
\newcommand{\velocityvecHat}{\hat{\velocityvec}}
\newcommand{\velocityvecOrigin}[1]{\velocityvec_{#1}}
\newcommand{\velocityvecHatOrigin}[1]{\velocityvecHat_{#1}}

\newcommand{\velocityvecDot}{\dot{\velocityvec}}
\newcommand{\velocityvecDotHat}{\dot{\velocityvecHat}}
\newcommand{\velocityvecDotOrigin}[1]{\dot{\velocityvecHat}_{#1}}
\newcommand{\velocityvecDotHatOrigin}[1]{\dot{\velocityvecHat}_{#1}}

\newcommand{\accelvec}{\mathbf{a}}
\newcommand{\accelvecHat}{\hat{\accelvec}}
\newcommand{\accelvecOrigin}[1]{\accelvec_{#1}}
\newcommand{\accelvecHatOrigin}[1]{\accelvecHat_{#1}}

\newcommand{\forcevec}{\mathbf{f}}
\newcommand{\forcevecHat}{\hat{\forcevec}}
\newcommand{\forcevecOrigin}[1]{\forcevec_{#1}}
\newcommand{\forcevecHatOrigin}[1]{\forcevecHat_{#1}}

\newcommand{\forcevecDot}{\dot{\forcevec}}
\newcommand{\forcevecDotHat}{\dot{\forcevecHat}}

\newcommand{\forcevecFrame}[1]{{^{#1}\mathbf{f}}}
\newcommand{\forcevecHatFrame}[1]{{^{#1}\hat{\forcevec}}}

\newcommand{\coord}[3]{{^#2\! #3\! _#1}}
\newcommand{\coordM}[4]{{^#2\! #3\! {_{#1}^{#4}}}}
\newcommand{\coordDual}[3]{{}^#2\! {#3}^{*}_{\! #1}}
\newcommand{\coordTrans}[3]{{}^#2\! {#3}^{T}_{\! #1}}
\newcommand{\coordInv}[3]{{}^#2\! {#3}^{-1}_{\! #1}}

\newcommand{\utransform}{X}
\newcommand{\transform}[2]{\coord{#1}{#2}{\utransform}}
\newcommand{\transformDual}[2]{\coordDual{#1}{#2}{\utransform}}
\newcommand{\transformTrans}[2]{\coordTrans{#1}{#2}{\utransform}}
\newcommand{\transformInv}[2]{\coordInv{#1}{#2}{\utransform}}
\newcommand{\transformM}[3]{\coordM{#1}{#2}{\utransform}{#3}}

\newcommand{\urotation}{E}
\newcommand{\rotation}[2]{\coord{#1}{#2}{\urotation}}
\newcommand{\rotationM}[3]{\coordM{#1}{#2}{\urotation}{#3}}

\newcommand{\utranslation}{r}
\newcommand{\translation}[2]{\coord{#1}{#2}{\utranslation}}
\newcommand{\translationM}[3]{\coordM{#1}{#2}{\utranslation}{#3}}

\newcommand{\inertia}{\mathbf{I}}
\newcommand{\inertiaBar}{\bar{\inertia}}
\newcommand{\inertiaBarOrigin}[1]{\inertiaBar_{#1}}

\newcommand{\momentum}{\mathbf{h}}
\newcommand{\momentumHat}{\hat{\momentum}}

\newcommand{\bodycom}{\mathbf{c}}
\newcommand{\mass}{\mathbf{m}}


\definecolor{listinggray}{gray}{0.9}
\definecolor{lbcolor}{rgb}{0.9,0.9,0.9}
\lstset{
backgroundcolor=\color{lbcolor},
	tabsize=2,
	%   rulecolor=,
	language=C++,
	basicstyle=\scriptsize,
	upquote=true,
	aboveskip={1.5\baselineskip},
	columns=fixed,
	showstringspaces=false,
	extendedchars=false,
	breaklines=true,
	prebreak = \raisebox{0ex}[0ex][0ex]{\ensuremath{\hookleftarrow}},
	frame=single,
	showtabs=false,
	showspaces=false,
	showstringspaces=false,
	identifierstyle=\ttfamily,
	keywordstyle=\color[rgb]{0,0,1},
	commentstyle=\color[rgb]{0.026,0.512,0.095},
	stringstyle=\color[rgb]{0.627,0.126,0.941},
	numberstyle=\color[rgb]{0.205, 0.142, 0.73},
	%        \lstdefinestyle{C++}{language=C++,style=numbers}’.
}


\begin{document}
	\frame{\titlepage}

	\begin{frame}
		\frametitle{Table of Contents}
		\tableofcontents
  	\end{frame}


	\section{Spatial Vector Algebra}
  	\begin{frame}
		\frametitle{Spatial Vector Algebra}
		\framesubtitle{What is it ?}
		Spatial vector algebra is a concise vector notation for describing rigid−body velocity,
		acceleration, inertia, etc., using 6D vectors and tensors.
		\begin{itemize}
			\item fewer quantities
			\item fewer equations
			\item less effort
			\item fewer mistakes
		\end{itemize}
  	\end{frame}


  	\begin{frame}
		\frametitle{Spatial Vector Algebra}
		\framesubtitle{Spatial vector spaces}
		There is 2 vector spaces:
		\begin{itemize}
			\item $ \motionspace{6} $ - motion vector (velocity, acceleration, ...)
			\item $ \forcespace{6} $ - force vector (momentum, force, ...)
		\end{itemize}
  	\end{frame}


  	\begin{frame}
		\frametitle{Spatial Vector Algebra}
		\framesubtitle{Spatial velocity vector}
		The spatial velocity of a point $ O $ of a body is:
		\begin{subequations}
			$$
			\velocityvecHatOrigin{O} = \begin{bmatrix} w_x \\ w_y \\ w_z \\ v_{Ox} \\ v_{Oy} \\ v_{Oz} \end{bmatrix} = \begin{bmatrix} w \\ v_O \end{bmatrix}
			$$
			$$
			\motionvecHatOrigin{O} \in \motionspace{6}
			$$
		\end{subequations}
		With the angular velocity:
		$ w = \begin{bmatrix} w_x\ w_y\ w_z \end{bmatrix}^T $\\
		And the linear velocity at O:
		$ v_O = \begin{bmatrix} v_{Ox}\ v_{Oy}\ v_{Oz} \end{bmatrix}^T $
	\end{frame}


  	\begin{frame}
		\frametitle{Spatial Vector Algebra}
		\framesubtitle{Spatial acceleration vector}
		The spatial acceleration of a point $ O $ of a body is:
		\begin{subequations}
			$$
			\accelvecHatOrigin{O} = \velocityvecDotHatOrigin{O} = \begin{bmatrix} \dot{w} \\ \dot{v}_O \end{bmatrix}
			$$
			$$
			\accelvecHatOrigin{O} \in \motionspace{6}
			$$
		\end{subequations}
		Beware that $ \dot{v}_O $ state for tangential acceleration and normal acceleration.
		If we define $ r_O $ the body O point coordinate at time $ t $, $ \dot{r}_O $ his derivative and $ \ddot{r}_O $ his acceleration, then $ \dot{v}_O = \ddot{r}_O - w \times \dot{r}_O $ 

		\hfill \\
		We will use $ \motionvecHat $ to describe a generic motion vector.
	\end{frame}


  	\begin{frame}
		\frametitle{Spatial Vector Algebra}
		\framesubtitle{Spatial force vector}
		The spatial force of a point $ O $ of a body is:
		\begin{subequations}
			$$
			\forcevecHatOrigin{O} = \begin{bmatrix} n_{Ox} \\ n_{Oy} \\ n_{Oz} \\ f_x \\ f_y \\ f_z \end{bmatrix} = \begin{bmatrix} n_O \\ f \end{bmatrix}
			$$
			$$
			\forcevecHatOrigin{O} \in \forcespace{6}
			$$
		\end{subequations}
		With the torque at point O:
		$ n_O = \begin{bmatrix} n_{Ox}\ n_{Oy}\ n_{Oz} \end{bmatrix}^T $\\
	        And the force:	
		$ f = \begin{bmatrix} f_x\ f_y\ f_z \end{bmatrix}^T $.
	\end{frame}


  	\begin{frame}
		\frametitle{Spatial Vector Algebra}
		\framesubtitle{Spatial transformations}
		The motion vector transformation from the frame A to B is written:
		$$
		\transform{A}{B} = \begin{bmatrix} \rotation{A}{B} & 0 \\ -\rotation{A}{B} \translation{A}{B} \times & \rotation{A}{B} \end{bmatrix}
		$$
		With $ \rotation{A}{B} \in \mathbb{R}^{3{\times}3} $ and $ \translation{A}{B} \in \mathbb{R}^{3} $ the A to B rotation matrix and translation vector.

		\hfill \\
		To apply the same transformation to a force vector, we must use the dual transform:
		$$
		\transformDual{A}{B} = (\transformInv{A}{B})^T
		$$
	\end{frame}


  	\begin{frame}
		\frametitle{Spatial Vector Algebra}
		\framesubtitle{Transformations example}
		Transform a motion vector in A frame $ \motionvecHatFrame{A} $ to B frame $ \forcevecHatFrame{B} $:
		$$
		\motionvecHatFrame{B} = \transform{A}{B} \motionvecHatFrame{A}
		$$

		Transform a force vector in A frame $ \forcevecHatFrame{A} $ to B frame $ \forcevecHatFrame{B} $:
		$$
		\forcevecHatFrame{B} = \transformDual{A}{B} \forcevecHatFrame{A}
		$$

		Find the transformation between A and C frame:
		$$
		\transform{A}{C} = \transform{B}{C}\transform{A}{B}
		$$
	\end{frame}


  	\begin{frame}
		\frametitle{Spatial Vector Algebra}
		\framesubtitle{Spatial rigid body inertia}
		The spatial inertia at the origin O of a body is:
		$$
		\inertia = \begin{bmatrix} \inertiaBarOrigin{O} & \mass \bodycom \times \\ -\mass \bodycom \times & \mass1 \end{bmatrix}
		$$
		With $ \inertiaBarOrigin{O} $ is the body inertia matrix at his origin O, $ \mass $ the body mass and $ \bodycom = \translation{O}{{CoM}} $ the translation between the body origin and his center of mass.

		\hfill \\
		This matrix allow to transform a $ \motionspace{6} $ in a $ \forcespace{6} $.
	\end{frame}


  	\begin{frame}
		\frametitle{Spatial Vector Algebra}
		\framesubtitle{Spatial inertia use}
		Transform an acceleration to a force:
		$$
		\forcevecHat = \inertia \accelvecHat
		$$
		A velocity to a spatial momentum:
		$$
		\momentumHat = \inertia \velocityvecHat
		$$
		Merge body $ b_2 $ inertia into body $ b_1 $ inertia:
		$$
		{}^{b_1+b_2}\inertia = {}^{b_1}\inertia + \transform{{b_2}}{{b_1}} {}^{b_2}\inertia\transformInv{{b_2}}{{b_1}}
		$$
	\end{frame}
	\section{SpaceVecAlg Library}

  	\begin{frame}
		\frametitle{SpaceVecAlg}
		\framesubtitle{What's in ?}
		\begin{itemize}
			\item Featherstone Spatial Vector Algebra C++11 implementation
			\item Header only
			\item Use Eigen3 as linear algebra library
			\item Python binding
		\end{itemize}
	\end{frame}


  	\begin{frame}[fragile]
		\frametitle{SpaceVecAlg}
		\framesubtitle{MotionVec}
		MotionVec is the Spatial Motion Vector implementation:
		\begin{lstlisting}[language=C++]
			Eigen::Vector3d w, v;

			sva::MotionVecd mv1(w, v); // constructor
			w == mv1.angular(); // angular getter
			v == mv1.linear(); // linear getter

			sva::MotionVecd mv2;
			mv1 + mv2; // addition
			mv1 - mv2; // substraction
			10.*mv1; // scalar multiplication
		\end{lstlisting}
	\end{frame}


  	\begin{frame}[fragile]
		\frametitle{SpaceVecAlg}
		\framesubtitle{ForceVec}
		ForceVec is the Spatial Force Vector implementation:
		\begin{lstlisting}[language=C++]
			Eigen::Vector3d t, f;

			sva::ForceVecd fv1(t, f); // constructor
			t == mv1.couple(); // couple getter
			f == mv1.force(); // force getter

			sva::ForceVecd fv2;
			fv1 + fv2; // addition
			fv1 - fv2; // substraction
			10.*fv1; // scalar multiplication
		\end{lstlisting}
	\end{frame}


  	\begin{frame}[fragile]
		\frametitle{SpaceVecAlg}
		\framesubtitle{RBInertia}
		RBInertia is the Spatial Rigid Body Inertia implementation:
		\begin{lstlisting}[language=C++]
			Eigen::Vector3d com; // orgin to CoM translation
			double mass; // rigid body mass
			Eigen::Vector3d h = com*mass; // first CoM moment
			Eigen::Matrix3d I; // rigid body inertia at origin

			sva::RBInertiad rbi1(mass, h, I); // constructor
			mass == rbi1.mass(); // mass getter
			h == rbi1.momentum(); // momentum getter
			I == rbi1.inertia(); // inertia getter

			sva::RBInertiad rbi2(mass, h, I); // constructor
			rbi1 + rbi2; // addition
			rbi1 - rbi2; // substraction
			10.*rbi1; // scalar multiplication (only on mass and h)

			sva::MotionVecd mv;
			sva::ForceVecd fv = rbi1*mv;
		\end{lstlisting}
	\end{frame}


  	\begin{frame}[fragile]
		\frametitle{SpaceVecAlg}
		\framesubtitle{PTransform}
		PTransform is the Spatial Transformation implementation:
		\begin{lstlisting}[language=C++]
			Eigen::Matrix3d E; // rotation
			Eigen::Quaterniond q; // rotation
			Eigen::Vector3d r; // translation

			sva::PTransformd X1(E,r); // constructors
			sva::PTransformd X2(q,r); // quaternion -> matrix
			sva::PTransformd X3 = sva::PTransformd::Identity();
			E == X1.rotation(); // rotation getter
			r == X1.translation(); // translation getter

			// inverse function
			E.transpose() == X1.inv().rotation();
			-E*r == X1.inv().translation();
		\end{lstlisting}
	\end{frame}
  	\begin{frame}[fragile]
		\frametitle{SpaceVecAlg}
		\framesubtitle{PTransform}
		\begin{lstlisting}[language=C++]
			// motion vector transform and inverse transform
			sva::MotionVecd mv;
			sva::MotionVecd mv1 = X1*mv;
			mv1 == X1.invMul(mv1);

			// force vector transform and inverse transform
			sva::ForceVecd fv;
			sva::ForceVecd fv1 = X.dualMul(fv);
			fv == X.transMul(fv1);

			// inertia transform and inverse transform
			sva::RBInertiad rbi;
			sva::RBinertia rbi1 = X.dualMul(rbi);
			rbi = X.transMul(rbi1);
		\end{lstlisting}
	\end{frame}


  	\begin{frame}[fragile]
		\frametitle{SpaceVecAlg}
		\framesubtitle{Utilities}
		Some useful functions:
		\begin{lstlisting}[language=C++]
			// Anti trigonometric rotation around 1 axis
			double theta;
			Eigen::Matrix3d Ex = sva::RotX(theta); // X rotation
			Eigen::Matrix3d Ey = sva::RotY(theta); // Y rotation
			Eigen::Matrix3d Ez = sva::RotZ(theta); // Z rotation

			// 3d projection of an rotation error
			// x, y, z rotation to go from Ex to Ey
			Eigen::Vector3d sva::rotationError(Ex, Ey);

			// compute inertia at origin from inertia at CoM
			Eigen::Matrix3d IatCoM, IatO;
			Eigen::Matrix3d E; // rotation from origin to com frame
			Eigen::Vector3d com; // translation from origin to com
			double mass;
			IatO = inertiaToOrigin(IatCoM, mass, com, E);
		\end{lstlisting}
	\end{frame}

	\section{Rigid Body System}

  	\begin{frame}
		\frametitle{Rigid Body System}
		\framesubtitle{Description}
		A rigid body system is composed of a set of body linked by joint.
		There is many kind of rigid body system:
		\begin{itemize}
			\item Kinematic chain
			\item Kinematic tree
			\item Closed loop kinematic tree
		\end{itemize}
		We will focus on kinematic tree.
	\end{frame}
  	\begin{frame}
		\frametitle{Rigid Body System}
		\framesubtitle{Body}
		A kinematic tree contains $ N_B $ body.

		\hfill \\
		A body $ i \in \{1, \cdots, N_B\} $ is associated to an inertia $ I_i $.

		\hfill \\
		The parent body index of a body $ i \in \{1, \cdots, N_B\} $ is give by the $ \lambda $ array.
	\end{frame}
  	\begin{frame}
		\frametitle{Rigid Body System}
		\framesubtitle{Joint}
		A kinematic tree contains $ N_B $ joint.

		\hfill \\
		A joint $ i \in \{1, \cdots, N_B\} $ support the body $ i $.

		\hfill \\
		$ jtype(i) $ identify the joint $ i \in \{1, \cdots, N_B\} $ type.

		\hfill \\
	        The transformation $ \utransform_T(i) $	identify the joint $ i \in \{1, \cdots, N_B\} $
		static transformation between his parent body $ \lambda(i) $ and the joint $ i $ origin.
	\end{frame}
  	\begin{frame}
		\frametitle{Rigid Body System}
		\framesubtitle{Joint type}
		Each joint are characterized by a type $ jtype $.

		\hfill \\
		This type allow us to size of his general position vector $ \mathbf{q} $ and the size
		of his general velocity vector $ \mathbf{\alpha} $.

		\hfill \\
		With the $ jtype $, $ \mathbf{q} $ and $ \mathbf{\alpha} $ we are able to compute the joint
		transformation $ \utransform_J $ , motion subspace matrix $ \mathbf{S} $ and
		the joint velocity $ \velocityvec_J $:
		$$
		[\utransform_J, \mathbf{S}, \velocityvec_J] = jcalc(jtype, \mathbf{q}, \mathbf{\alpha})
		$$

		We can compute the joint velocity with the motion subspace matrix and
		the general velocity vector $ \velocityvec_J = \mathbf{S}\mathbf{\alpha} $.
		That allow to compute the kinematics tree Jacobian really easily.
	\end{frame}



	\section{RBDyn Library}
  	\begin{frame}
		\frametitle{RBDyn}
		\framesubtitle{Description}
		Resume:
		\begin{itemize}
			\item Kinematics tree Kinematics and Dynamics algorithm C++11 implementation
			\item Use Eigen3 and SpaceVecAlg library
			\item Free, Spherical, Planar, Cylindrical, Revolute, Prismatic joint support
			\item Translation, Rotation, Vector, CoM, Momentum Jacobian computation
			\item Inverse dynamics, Forward dynamics
			\item Kinematics tree body merging/filtering
			\item Kinematics tree base selection
			\item Python binding
		\end{itemize}
	\end{frame}

  	\begin{frame}[fragile]
		\frametitle{RBDyn}
		\framesubtitle{Body}
		A body is constituted of a spatial inertia, and is identified by an unique id and name:
		\begin{lstlisting}[language=C++]
			sva::RBInertiad rbi; // body ri
			Eigen::Vector3d r_com;
			Eigen::Matrix3d I;
			double mass;
			int id;
			std::string name;

			// constructors
			rbd::Body b1(rbi, id, name);
			rbd::Body b2(mass, r_com, I, id, name);

			rbi == b1.inertia(); // inertia getter
			id == b1.id(); // id getter
			name == bi1.name(); // name getter
		\end{lstlisting}
	\end{frame}

  	\begin{frame}[fragile]
		\frametitle{RBDyn}
		\framesubtitle{Joint}
		A joint is constituted of a type, an optional axis, a direction and like the body is identified by an
		unique id and name:
		\begin{lstlisting}[language=C++]
			Joint::Type type; // Rev, Prism, Spherical, Planar, Cylindrical, Free and Fixed
			Eigen::Vector3d axis; // For Rev, Prism and Cylindrical
			bool direction; // true forward, false backward
			int id;
			std::string name;

			// constructors
			rbd::Joint j1(type, axis, direction, id, name);
			rbd::Joint j2(type, direction, id, name);

			type == j1.type(); // type getter
			direction == j1.forward(); // direction getter
			id == j1.id(); // id getter
			name == j1.name(); // name gette
		\end{lstlisting}
	\end{frame}
  	\begin{frame}[fragile]
		\frametitle{RBDyn}
		\framesubtitle{Joint}
		\begin{lstlisting}[language=C++]
			j1.params(); // q vector size
			j1.dof(); // alpha vector size

			j1.motionSubsace(); // motion subspace matrix (S)

			// transformation and velocity
			Eigen::VectorXd q, alpha, alphaDot;
			j1.pose(q); // X_j transformation matrix
			j1.motion(alpha); // v_j motion vector
			j1.tanAccel(alphaDot); // tangential acceleration

			// initialization
			j1.zeroParam(); // identity q vector
			j1.zeroDof); // identity alpha, alphaDot vector
		\end{lstlisting}
	\end{frame}

  	\begin{frame}[fragile]
		\frametitle{RBDyn}
		\framesubtitle{MultiBodyGraph}
		A non oriented graph that model the robot:
		\begin{lstlisting}[language=C++]
			sva::RBInertiad rbi;
			rbd::Body b1(rbi, 1, "b1"), b2(rbi, 2, "b2"), b3(rbi, 3, "b3");
			rbd::Joint j1(rbd::Joint::Rev, Eigen::Vector3d::UnitX, true, 1, "j1");
			rbd::Joint j2(rbd::Joint::Spherical, true, 2, "j2");
			sva::PTransformd X_12, X_13;
			sva::PTransformd I(sva::PTransformd::Identity());

			rbd::MultiBodyGraph mbg; // constructor
			mbg.addBody(b1);mbg.addBody(b2);mbg.addBody(b3);
			mbg.addJoint(j12);mbg.addJoint(j13);

			// link body b1 to body b2 with joint j1 and static transform X_12
			mbg.linkBodies(1, X_12, 2, I, 1);
			// link body b1 to body b3 with joint j2 and static transform X_13
			mbg.linkBodies(1, X_13, 3, I, 2);
		\end{lstlisting}
	\end{frame}
  	\begin{frame}[fragile]
		\frametitle{RBDyn}
		\framesubtitle{MultiBodyGraph}
		\begin{lstlisting}[language=C++]
			// create a multibody with b1 as fixed root body
			rbd::MultiBody mb1 = mbg.makeMultiBody(1, rbd::Joint::Fixed);

			sva::PTransformd X_O_j0, X_b0_j0;
			// multibody with b2 as planar base 
			// X_O_j0 is the transform from the world to the planar joint
			// X_b0_j0 is the transform from b2 to the joint (X_T)
			rbd::MultiBody mb2 = mbg.makeMultiBody(2, rbd::Joint::Planar, X_O_j0, X_b0_j0);

			// remove a joint and his subtree
			mbg.removeJoint(...);

			// merge a sub tree in one body
			mbg.mergeSubBodies(...);

			// compute bodies transformation to them origin
			mbg.bodiesBaseTransform(...);
		\end{lstlisting}
	\end{frame}
  	\begin{frame}
		\frametitle{RBDyn}
		\framesubtitle{MultiBodyGraph illustration}
	\end{frame}

	\begin{frame}[fragile]
		\frametitle{RBDyn}
		\framesubtitle{MultiBody}
		Kinematics tree implementation:
		\begin{lstlisting}[language=C++]
			mb1.nrBodies() == 3; // Body number
			mb1.nrJonits() == 3; // Joint number

			mb1.bodies(); // bodies array
			mb1.joints(); // joints array

			// body that pred/succ a joint
			mb1.predecessors();
			mb1.successors();
			mb1.parents(); // lambda
			mb1.transfroms(); // X_T array

			// body/joint index by id
			mb1.bodyIndexById(id);
			mb1.jointIndexById(id);

			mb1.bodyIndexById(1) != mb2.bodyIndexById(1);
		\end{lstlisting}
	\end{frame}
	\begin{frame}[fragile]
		\frametitle{RBDyn}
		\framesubtitle{MultiBody}
		\begin{lstlisting}[language=C++]
			//                     Spherical   Rev   Fixed
			mb1.nrParams() == 5; //   4         1      0
			mb1.nrDof() == 4; //      3         1      0

			//                     Spherical   Rev   Planar
			mb2.nrParams() == 8; //   4         1      3
			mb2.nrDof() == 7; //      3         1      3

			mb1.jointPosInParam(index) // joint position in q
			mb1.jointPosInDof(index) // joint position in alpha
		\end{lstlisting}
	\end{frame}

	\begin{frame}[fragile]
		\frametitle{RBDyn}
		\framesubtitle{MultiBodyConfig}
		Configuration of a multi body is separated from his model:
		\begin{lstlisting}[language=C++]
			rbd::MultiBodyConfig mbc1(mb1);
			mbc1.zero(mb1); // set all variable to zero

			// indexed by joint
			// std::vector<std::vector<double>>
			mbc1.q; // q generalized position vector
			mbc1.alpha; // alpha generalized velocity vector
			mbc1.alphaD; // q generalized acceleration vector
			mbc1.jointTorque; // joint torque

			// indexed by body
			mbc1.force; // Force apply on each body
			mbc1.bodyPosW; // Body transformation in world coord
			mbc1.bodyVelW; // Body velocity in world coord
			mbc1.bodyVelB; // Body velocity in body coord

			mbc.gravity; // gravity apply on the system
			...
		\end{lstlisting}
	\end{frame}


	\begin{frame}[fragile]
		\frametitle{RBDyn}
		\framesubtitle{MultiBodyConfig usefull functions}
		\begin{lstlisting}[language=C++]
			// convert q, alpha, alphaD and force vector
			// from mbc1 to mbc2
			// mbc1 and mbc2 must come from the same mbg
			// don't convert root joint
			rbd::ConfigConverter(mbc1, mbc2);

			Eigen::VectorXd q;
			// to and from eigen3 vector
			paramToVector(mbc.q, q);
			vectorToParam(q, mbc.q);
		\end{lstlisting}
	\end{frame}

	\begin{frame}[fragile]
		\frametitle{RBDyn}
		\framesubtitle{Algorithm}
		\begin{lstlisting}[language=C++]
			// q -> bodyPosW
			rbd::forwardKinematics(mb1, mbc1);

			// alpha, FK -> bodyVelW, bodyVelB
			rbd::forwardVelocity(mb1, mbc1);

			// alphaD, gravity, FV -> jointTorque
			rbd::InverseDynamics id(mb1);
			id.inverseDynamics(mb1, mbc1);

			// jointTorque, FV -> alphaD, H, C
			rbd::ForwardDynamics fd(mb1);
			fd.forwardDynamics(mb1, mbc1);

			// FK -> CoM
			rbd::computeCoM(mb1, mbc1);

			// FV -> CoM speed
			rbd::computeCoMVelocity(mb1, mbc1);
		\end{lstlisting}
	\end{frame}

	\begin{frame}[fragile]
		\frametitle{RBDyn}
		\framesubtitle{Jacobian}
		\begin{lstlisting}[language=C++]
			Eigen::Vector3d bodyPoint;
			rbd::Jacobian jac(mb1, bodyId, bodyPoint);

			// 6D bodyId world coord jacobian at bodyPoint
			jac.jacobian(mb1, mbc1);
			// bodyId coordinate
			jac.bodyJacobian(mb1, mbc1);

			Eigen::Vector3d vec;
			// 3D world coord jacobian of a vector in bodyId
			jac.vectorJacobian(mb1, mbc1, vec);
			// bodyId coordinate
			jac.bodyVectorJacobian(mb1, mbc1, vec);

			// translate a jacobian to a specfic point
			jac.translateJacobian(...);

			// project a jacobian in the robot parameters vector
			jac.fullJacobian(...);
		\end{lstlisting}
	\end{frame}

	\begin{frame}
		\frametitle{RBDyn}
		\framesubtitle{Other stuff}
		\begin{itemize}
			\item $ \dot{J} $ computation
			\item Jacobian normal acceleration vector $ \dot{J}\dot{\mathbf{q}} $ computation
			\item CoM Zacobian
			\item Centroidal Momentum Matrix $ A_g $ and $ \dot{A}_g $
			\item Centroidal ZMP
			\item Euler integration
		\end{itemize}
	\end{frame}

	\begin{frame}
		\frametitle{Next time}
		\framesubtitle{}
		We will see how to use SpaceVecAlg and RBDyn in Python with ROS.
		
		\hfill \\
		In the same time we will learn how to use Tasks library to do whole body control on an humanoid robot.

		\hfill \\
		{\centering
		Stay tuned !\par}
		\hfil \\
		{\centering
		...\par}
		\hfill \\
		{\centering
		\bf{Questions} ?\par}
	\end{frame}
\end{document}

