\documentclass{beamer}

\usetheme{PaloAlto}
\usecolortheme{crane}

\title
{A short introduction to SVA and RBDyn}
\author
{Joris Vaillant}
\institute{LIRMM}{}
\date{Thursday 4 2014}
\subject{Spatial Vector Algebra, Rigid Body Physics}

\usepackage{listings}
\usepackage{graphicx}      % include this line if your document contains figures
\usepackage{amsmath} % assumes amsmath package installed
\usepackage{amsfonts} %% pour mathbb

% vim: set fileencoding=utf8 :

\newcommand{\euclidspace}[1]{E^{#1}}
\newcommand{\motionspace}[1]{M^{#1}}
\newcommand{\forcespace}[1]{F^{#1}}

\newcommand{\motionvec}{\mathbf{m}}
\newcommand{\motionvecHat}{\hat{\motionvec}}
\newcommand{\motionvecOrigin}[1]{\motionvec_{#1}}
\newcommand{\motionvecHatOrigin}[1]{\motionvecHat_{#1}}

\newcommand{\motionvecDot}{\dot{\motionvec}}
\newcommand{\motionvecDotHat}{\dot{\motionvecHat}}

\newcommand{\motionvecFrame}[1]{{^{#1}\mathbf{m}}}
\newcommand{\motionvecHatFrame}[1]{{^{#1}\hat{\motionvec}}}

\newcommand{\velocityvec}{\mathbf{v}}
\newcommand{\velocityvecHat}{\hat{\velocityvec}}
\newcommand{\velocityvecOrigin}[1]{\velocityvec_{#1}}
\newcommand{\velocityvecHatOrigin}[1]{\velocityvecHat_{#1}}

\newcommand{\velocityvecDot}{\dot{\velocityvec}}
\newcommand{\velocityvecDotHat}{\dot{\velocityvecHat}}
\newcommand{\velocityvecDotOrigin}[1]{\dot{\velocityvecHat}_{#1}}
\newcommand{\velocityvecDotHatOrigin}[1]{\dot{\velocityvecHat}_{#1}}

\newcommand{\accelvec}{\mathbf{a}}
\newcommand{\accelvecHat}{\hat{\accelvec}}
\newcommand{\accelvecOrigin}[1]{\accelvec_{#1}}
\newcommand{\accelvecHatOrigin}[1]{\accelvecHat_{#1}}

\newcommand{\forcevec}{\mathbf{f}}
\newcommand{\forcevecHat}{\hat{\forcevec}}
\newcommand{\forcevecOrigin}[1]{\forcevec_{#1}}
\newcommand{\forcevecHatOrigin}[1]{\forcevecHat_{#1}}

\newcommand{\forcevecDot}{\dot{\forcevec}}
\newcommand{\forcevecDotHat}{\dot{\forcevecHat}}

\newcommand{\forcevecFrame}[1]{{^{#1}\mathbf{f}}}
\newcommand{\forcevecHatFrame}[1]{{^{#1}\hat{\forcevec}}}

\newcommand{\coord}[3]{{^#2\! #3\! _#1}}
\newcommand{\coordM}[4]{{^#2\! #3\! {_{#1}^{#4}}}}
\newcommand{\coordDual}[3]{{}^#2\! {#3}^{*}_{\! #1}}
\newcommand{\coordTrans}[3]{{}^#2\! {#3}^{T}_{\! #1}}
\newcommand{\coordInv}[3]{{}^#2\! {#3}^{-1}_{\! #1}}

\newcommand{\utransform}{X}
\newcommand{\transform}[2]{\coord{#1}{#2}{\utransform}}
\newcommand{\transformDual}[2]{\coordDual{#1}{#2}{\utransform}}
\newcommand{\transformTrans}[2]{\coordTrans{#1}{#2}{\utransform}}
\newcommand{\transformInv}[2]{\coordInv{#1}{#2}{\utransform}}
\newcommand{\transformM}[3]{\coordM{#1}{#2}{\utransform}{#3}}

\newcommand{\urotation}{E}
\newcommand{\rotation}[2]{\coord{#1}{#2}{\urotation}}
\newcommand{\rotationM}[3]{\coordM{#1}{#2}{\urotation}{#3}}

\newcommand{\utranslation}{r}
\newcommand{\translation}[2]{\coord{#1}{#2}{\utranslation}}
\newcommand{\translationM}[3]{\coordM{#1}{#2}{\utranslation}{#3}}

\newcommand{\inertia}{\mathbf{I}}
\newcommand{\inertiaBar}{\bar{\inertia}}
\newcommand{\inertiaBarOrigin}[1]{\inertiaBar_{#1}}

\newcommand{\momentum}{\mathbf{h}}
\newcommand{\momentumHat}{\hat{\momentum}}

\newcommand{\bodycom}{\mathbf{c}}
\newcommand{\mass}{\mathbf{m}}


\definecolor{listinggray}{gray}{0.9}
\definecolor{lbcolor}{rgb}{0.9,0.9,0.9}
\lstset{
backgroundcolor=\color{lbcolor},
	tabsize=2,
	%   rulecolor=,
	language=C++,
	basicstyle=\scriptsize,
	upquote=true,
	aboveskip={1.5\baselineskip},
	columns=fixed,
	showstringspaces=false,
	extendedchars=false,
	breaklines=true,
	prebreak = \raisebox{0ex}[0ex][0ex]{\ensuremath{\hookleftarrow}},
	frame=single,
	showtabs=false,
	showspaces=false,
	showstringspaces=false,
	identifierstyle=\ttfamily,
	keywordstyle=\color[rgb]{0,0,1},
	commentstyle=\color[rgb]{0.026,0.512,0.095},
	stringstyle=\color[rgb]{0.627,0.126,0.941},
	numberstyle=\color[rgb]{0.205, 0.142, 0.73},
	%        \lstdefinestyle{C++}{language=C++,style=numbers}’.
}


\begin{document}
	\frame{\titlepage}

	\begin{frame}
		\frametitle{Table of Contents}
		\tableofcontents
  	\end{frame}


	\section{Spatial Vector Algebra}
  	\begin{frame}
		\frametitle{Spatial Vector Algebra}
		\framesubtitle{What is it ?}
		Spatial vector algebra is a concise vector notation for describing rigid−body velocity,
		acceleration, inertia, etc., using 6D vectors and tensors.
		\begin{itemize}
			\item fewer quantities
			\item fewer equations
			\item less effort
			\item fewer mistakes
		\end{itemize}
  	\end{frame}


  	\begin{frame}
		\frametitle{Spatial Vector Algebra}
		\framesubtitle{Spatial vector spaces}
		There is 2 vector spaces:
		\begin{itemize}
			\item $ \motionspace{6} $ - motion vector (velocity, acceleration, ...)
			\item $ \forcespace{6} $ - force vector (momentum, force, ...)
		\end{itemize}
  	\end{frame}


  	\begin{frame}
		\frametitle{Spatial Vector Algebra}
		\framesubtitle{Spatial velocity vector}
		The spatial velocity of a point $ O $ of a body is:
		\begin{subequations}
			$$
			\velocityvecHatOrigin{O} = \begin{bmatrix} w_x \\ w_y \\ w_z \\ v_{Ox} \\ v_{Oy} \\ v_{Oz} \end{bmatrix} = \begin{bmatrix} w \\ v_O \end{bmatrix}
			$$
			$$
			\motionvecHatOrigin{O} \in \motionspace{6}
			$$
		\end{subequations}
		With the angular velocity:
		$ w = \begin{bmatrix} w_x\ w_y\ w_z \end{bmatrix}^T $\\
		And the linear velocity at O:
		$ v_O = \begin{bmatrix} v_{Ox}\ v_{Oy}\ v_{Oz} \end{bmatrix}^T $
	\end{frame}


  	\begin{frame}
		\frametitle{Spatial Vector Algebra}
		\framesubtitle{Spatial acceleration vector}
		The spatial acceleration of a point $ O $ of a body is:
		\begin{subequations}
			$$
			\accelvecHatOrigin{O} = \velocityvecDotHatOrigin{O} = \begin{bmatrix} \dot{w} \\ \dot{v}_O \end{bmatrix}
			$$
			$$
			\accelvecHatOrigin{O} \in \motionspace{6}
			$$
		\end{subequations}
		Beware that $ \dot{v}_O $ state for tangential acceleration and normal acceleration.
		If we define $ r_O $ the body O point coordinate at time $ t $, $ \dot{r}_O $ his derivative and $ \ddot{r}_O $ his acceleration, then $ \dot{v}_O = \ddot{r}_O - w \times \dot{r}_O $ 

		\hfill \\
		We will use $ \motionvecHat $ to describe a generic motion vector.
	\end{frame}


  	\begin{frame}
		\frametitle{Spatial Vector Algebra}
		\framesubtitle{Spatial force vector}
		The spatial force of a point $ O $ of a body is:
		\begin{subequations}
			$$
			\forcevecHatOrigin{O} = \begin{bmatrix} n_{Ox} \\ n_{Oy} \\ n_{Oz} \\ f_x \\ f_y \\ f_z \end{bmatrix} = \begin{bmatrix} n_O \\ f \end{bmatrix}
			$$
			$$
			\forcevecHatOrigin{O} \in \forcespace{6}
			$$
		\end{subequations}
		With the torque at point O:
		$ n_O = \begin{bmatrix} n_{Ox}\ n_{Oy}\ n_{Oz} \end{bmatrix}^T $\\
	        And the force:	
		$ f = \begin{bmatrix} f_x\ f_y\ f_z \end{bmatrix}^T $.
	\end{frame}


  	\begin{frame}
		\frametitle{Spatial Vector Algebra}
		\framesubtitle{Spatial transformations}
		The motion vector transformation from the frame A to B is written:
		$$
		\transform{A}{B} = \begin{bmatrix} \rotation{A}{B} & 0 \\ -\rotation{A}{B} \translation{A}{B} \times & \rotation{A}{B} \end{bmatrix}
		$$
		With $ \rotation{A}{B} \in \mathbb{R}^{3{\times}3} $ and $ \translation{A}{B} \in \mathbb{R}^{3} $ the A to B rotation matrix and translation vector.

		\hfill \\
		To apply the same transformation to a force vector, we must use the dual transform:
		$$
		\transformDual{A}{B} = (\transformInv{A}{B})^T
		$$
	\end{frame}


  	\begin{frame}
		\frametitle{Spatial Vector Algebra}
		\framesubtitle{Transformations example}
		Transform a motion vector in A frame $ \motionvecHatFrame{A} $ to B frame $ \forcevecHatFrame{B} $:
		$$
		\motionvecHatFrame{B} = \transform{A}{B} \motionvecHatFrame{A}
		$$

		Transform a force vector in A frame $ \forcevecHatFrame{A} $ to B frame $ \forcevecHatFrame{B} $:
		$$
		\forcevecHatFrame{B} = \transformDual{A}{B} \forcevecHatFrame{A}
		$$

		Find the transformation between A and C frame:
		$$
		\transform{A}{C} = \transform{B}{C}\transform{A}{B}
		$$
	\end{frame}


  	\begin{frame}
		\frametitle{Spatial Vector Algebra}
		\framesubtitle{Spatial rigid body inertia}
		The spatial inertia at the origin O of a body is:
		$$
		\inertia = \begin{bmatrix} \inertiaBarOrigin{O} & \mass \bodycom \times \\ -\mass \bodycom \times & \mass1 \end{bmatrix}
		$$
		With $ \inertiaBarOrigin{O} $ is the body inertia matrix at his origin O, $ \mass $ the body mass and $ \bodycom = \translation{O}{{CoM}} $ the translation between the body origin and his center of mass.

		\hfill \\
		This matrix allow to transform a $ \motionspace{6} $ in a $ \forcespace{6} $.
	\end{frame}


  	\begin{frame}
		\frametitle{Spatial Vector Algebra}
		\framesubtitle{Spatial inertia use}
		Transform an acceleration to a force:
		$$
		\forcevecHat = \inertia \accelvecHat
		$$
		A velocity to a spatial momentum:
		$$
		\momentumHat = \inertia \velocityvecHat
		$$
		Merge body $ b_2 $ inertia into body $ b_1 $ inertia:
		$$
		{}^{b_1+b_2}\inertia = {}^{b_1}\inertia + \transform{{b_2}}{{b_1}} {}^{b_2}\inertia\transformInv{{b_2}}{{b_1}}
		$$
	\end{frame}
	\section{SpaceVecAlg Library}

  	\begin{frame}
		\frametitle{SpaceVecAlg}
		\framesubtitle{What's in ?}
		\begin{itemize}
			\item Featherstone Spatial Vector Algebra C++11 implementation
			\item Header only
			\item Use Eigen3 as linear algebra library
			\item Python binding
		\end{itemize}
	\end{frame}


  	\begin{frame}[fragile]
		\frametitle{SpaceVecAlg}
		\framesubtitle{MotionVec}
		MotionVec is the Spatial Motion Vector implementation:
		\begin{lstlisting}[language=C++]
			Eigen::Vector3d w, v;

			sva::MotionVec mv1(w, v); // constructor
			w == mv1.angular(); // angular getter
			v == mv1.linear(); // linear getter

			sva::MotionVec mv2;
			mv1 + mv2; // addition
			mv1 - mv2; // substraction
			10.*mv1; // scalar multiplication
		\end{lstlisting}
	\end{frame}


  	\begin{frame}[fragile]
		\frametitle{SpaceVecAlg}
		\framesubtitle{ForceVec}
		ForceVec is the Spatial Force Vector implementation:
		\begin{lstlisting}[language=C++]
			Eigen::Vector3d t, f;

			sva::ForceVec fv1(t, f); // constructor
			t == mv1.couple(); // couple getter
			f == mv1.force(); // force getter

			sva::ForceVec fv2;
			fv1 + fv2; // addition
			fv1 - fv2; // substraction
			10.*fv1; // scalar multiplication
		\end{lstlisting}
	\end{frame}


  	\begin{frame}[fragile]
		\frametitle{SpaceVecAlg}
		\framesubtitle{RBInertia}
		RBInertia is the Spatial Rigid Body Inertia implementation:
		\begin{lstlisting}[language=C++]
			Eigen::Vector3d com; // orgin to CoM translation
			double mass; // rigid body mass
			Eigen::Vector3d h = com*mass; // first CoM moment
			Eigen::Matrix3d I; // rigid body inertia at origin

			sva::RBInertia rbi1(mass, h, I); // constructor
			mass == rbi1.mass(); // mass getter
			h == rbi1.momentum(); // momentum getter
			I == rbi1.inertia(); // inertia getter

			sva::RBInertia rbi2(mass, h, I); // constructor
			rbi1 + rbi2; // addition
			rbi1 - rbi2; // substraction
			10.*rbi1; // scalar multiplication (only on mass and h)

			sva::MotionVec mv;
			sva::ForceVec fv = rbi1*mv;
		\end{lstlisting}
	\end{frame}


  	\begin{frame}[fragile]
		\frametitle{SpaceVecAlg}
		\framesubtitle{PTransform}
		PTransform is the Spatial Transformation implementation;
		\begin{lstlisting}[language=C++]
			Eigen::Matrix3d E; // rotation
			Eigen::Quaterniond q; // rotation
			Eigen::Vector3d r; // translation

			sva::PTransform X1(E,r); // constructors
			sva::PTransform X2(q,r); // quaternion -> matrix
			sva::PTransform X3 = sva::PTransform::Identity();
			E == X1.rotation(); // rotation getter
			r == X1.translation(); // translation getter

			// inverse function
			E.transpose() == X1.inv().rotation();
			-E*r == X1.inv().translation();
		\end{lstlisting}
	\end{frame}
  	\begin{frame}[fragile]
		\frametitle{SpaceVecAlg}
		\framesubtitle{PTransform}
		\begin{lstlisting}[language=C++]
			// motion vector transform and inverse transform
			sva::MotionVec mv;
			sva::MotionVec mv1 = X1*mv;
			mv1 == X1.invMul(mv1);

			// force vector transform and inverse transform
			sva::ForceVec fv;
			sva::ForceVec fv1 = X.dualMul(fv);
			fv == X.transMul(fv1);

			// inertia transform and inverse transform
			sva::RBInertia rbi;
			sva::RBinertia rbi1 = X.dualMul(rbi);
			rbi = X.transMul(rbi1);
		\end{lstlisting}
	\end{frame}


  	\begin{frame}[fragile]
		\frametitle{SpaceVecAlg}
		\framesubtitle{Utilities}
		Some usefull functions:
		\begin{lstlisting}[language=C++]
			// Anti trigonometric rotation around 1 axis
			Eigen::Matrix3d Ex = sva::RotX(theta); // X rotation
			Eigen::Matrix3d Ey = sva::RotY(theta); // Y rotation
			Eigen::Matrix3d Ez = sva::RotZ(theta); // Z rotation

			// 3d projection of an rotation error
			// x, y, z rotation to go from Ex to Ey
			Eigen::Vector3d sva::rotationError(Ex, Ey);

			// compute inertia at origin from inertia at CoM
			Eigen::Matrix3d IatCoM, IatO;
			Eigen::Matrix3d E; // rotation from origin to com frame
			Eigen::Vector3d com; // translation from origin to com
			double mass;
			IatO = inertiaToOrigin(IatCoM, mass, com, E);
		\end{lstlisting}
	\end{frame}

\end{document}

